本章中,描述了统一共享内存,一种基于指针的数据管理策略。介绍了USM定义的三种类型的分配。讨论了使用USM分配和释放内存的所有不同方式,以及数据移动可以由开发者显式控制进行设备分配,也可以由系统隐式控制共享分配。最后,讨论了如何查询设备支持的不同USM功能,以及如何在程序中查询关于USM指针的信息。\par

因为还没有在本书中详细讨论同步,所以在后面的章节中会讨论调度、通信和同步的时候会有更多关于USM的内容。具体地说,将在第8、9和19章中继续讨论USM。\par

在下一章中,我们将介绍数据管理的第二种策略:缓冲区。\par

\newpage