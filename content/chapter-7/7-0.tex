\begin{center}
	\includegraphics[width=0.3\textwidth]{content/chapter-7/images/1}
\end{center}

In this chapter, we will learn about the buffer abstraction. We learned about Unified Shared Memory (USM), the pointer-based strategy for data management, in the previous chapter. USM forces us to think about where memory lives and what should be accessible where. The buffer abstraction is a higher-level model that hides this from the programmer. Buffers simply represent data, and it becomes the job of the runtime to manage how the data is stored and moved in memory.\par

This chapter presents an alternative approach to managing our data. The choice between buffers and USM often comes down to personal preference and the style of existing code, and applications are free to mix and match the two styles in representation of different data within the application.\par

USM simply exposes different abstractions for memory. USM has pointers, and buffers are a higher-level abstraction. The abstraction level of buffers allows the data contained within to be used on any device within the application, where the runtime manages whatever is needed to make that data available. Choices are good, so let’s dive into buffers.\par

We will look more closely at how buffers are created and used. A discussion of buffers would not be complete without also discussing the accessor. While buffers abstract how we represent and store data in a program, we do not directly access the data using the buffer. Instead, we use accessor objects that inform the runtime how we intend to use the data we are accessing, and accessors are tightly coupled to the powerful data dependence mechanisms within task graphs. After we cover all the things we can do with buffers, we will also explore how to create and use accessors in our programs.\par

























