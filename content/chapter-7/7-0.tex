\begin{center}
	\includegraphics[width=0.3\textwidth]{content/chapter-7/images/1}
\end{center}

在本章中,我们将了解缓冲区。在前一章了解了基于指针的数据管理策略USM (Unified Shared Memory, USM)。USM使我们思考内存存在于哪里,以及在哪里应该访问什么。缓冲区抽象是一个高级模型,它向开发者隐藏了这一点。缓冲区只是表示数据,而管理数据在内存中存储和移动的方式就成了运行时的工作。\par

本章介绍了一种管理数据的替代方法。缓冲区和USM之间的选择通常取决于个人偏好和现有代码的风格,应用程序可以自由地混合和匹配两种风格来表示应用程序中的不同数据。\par

USM只是公开了内存的不同抽象。USM有指针,而缓冲区是更高层次的抽象。缓冲区的抽象级别允许在应用程序的任何设备上,使用包含在其中的数据,运行时管理使该数据必须可用。\par

我们将更详细地了解缓冲区是如何创建和使用的。如果不讨论访问器,对缓冲区的讨论就不完整。虽然缓冲区抽象了程序中表示和存储数据的方式,但我们不使用缓冲区直接访问数据。相反,我们使用访问器对象告知运行时我们打算如何使用正在访问的数据,并且访问器与任务图中强大的数据依赖机制紧密耦合。在介绍了使用缓冲区可以做的所有事情之后,还将探讨如何在程序中创建和使用访问器。\par

























