命名函数对象,也称为函子,允许在定义良好的接口的同时操作任意数据集合。当用于表示内核时,已命名函数对象的成员变量定义了内核可能操作的状态,重载函数调用operator()将为并行执行的每个工作项调用。\par

命名函数对象需要比Lambda表达式更多的代码来表示内核,但是额外的控制提供了额外的功能。例如,更容易分析和优化以命名函数对象表示的内核,因为内核使用的任何缓冲区和数据值都必须显式地传递给内核,而不是自动捕获。\par

最后,因为命名函数对象就像任何其他C++类一样,表示为命名函数对象的内核可以模板化。以命名函数对象表示的内核也更容易重用,可以作为单独头文件或库的一部分。\par

\hspace*{\fill} \par %插入空行
\textbf{内核命名函数对象的组件}

图10-6中的代码描述了用命名函数对象表示的内核。\par

\hspace*{\fill} \par %插入空行
图10-6 内核作为命名函数对象
\begin{lstlisting}[caption={}]
class Add {
public:
	Add(accessor<int> acc) : data_acc(acc) {}
	void operator()(id<1> i) {
		data_acc[i] = data_acc[i] + 1;
	}

private:
	accessor<int> data_acc;
};

int main() {
	constexpr size_t size = 16;
	std::array<int, size> data;
	
	for (int i = 0; i < size; i++)
		data[i] = i;
		
	{
		buffer data_buf{data};
		
		queue Q{ host_selector{} };
		std::cout << "Running on device: "
				  << Q.get_device().get_info<info::device::name>() << "\n";
				  
		Q.submit([&](handler& h) {
			accessor data_acc {data_buf, h};
			h.parallel_for(size, Add(data_acc));
		});
	}
});
\end{lstlisting}

当内核表示为命名函数对象时,必须遵循C++11规则才能复制。这意味着命名函数对象可以安全地一个字节一个字节地复制,使命名函数对象的成员变量能够传递给在设备上执行的内核代码。\par

重载函数调用operator()的参数取决于内核,就像用Lambda表达式表示的内核一样。\par

因为函数对象是命名的,所以宿主代码编译器可以使用函数对象类型与设备代码编译器生成的内核代码相关联。因此,命名内核函数对象不需要额外的内核名称。\par












