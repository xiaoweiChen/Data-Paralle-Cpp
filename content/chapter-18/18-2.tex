DPC++库由以下组件组成:\par

\begin{itemize}
	\item 经过测试的C++标准API——只需要包含相应的C++标准头文件并使用std命名空间。
	\item 包含相应头文件使用并行STL。只需使用\#include <dpstd …>来包含它们。DPC++库为扩展的API类和函数使用名称空间dpstd。
\end{itemize}

\hspace*{\fill} \par %插入空行
\textbf{DPC++中的标准C++ API}

DPC++库包含一组经过测试的标准C++ API。许多C++标准API的基本功能已经开发,因此这些API可以在设备内核中使用,就像在C++主机应用程序的代码中使用一样。图18-7显示了如何在设备代码中使用std::swap的示例。\par

\hspace*{\fill} \par %插入空行
图18-7 在设备代码中使用std::swap
\begin{lstlisting}[caption={}]
class KernelSwap;
std::array <int,2> arr{8,9};
buffer<int> buf{arr};

{
	host_accessor host_A(buf);
	std::cout << "Before: " << host_A[0] << ", " << host_A[1] << "\n";
} // End scope of host_A so that upcoming kernel can operate on buf

queue Q;
Q.submit([&](handler &h) {
	accessor A{buf, h};
	h.single_task([=]() {
		// Call std::swap!
		std::swap(A[0], A[1]);
	});
});

host_accessor host_B(buf);
std::cout << "After: " << host_B[0] << ", " << host_B[1] << "\n";
\end{lstlisting}

可以使用以下命令来构建和运行程序(假设它位于stdswap.cpp文件中):\par

\begin{tcolorbox}[colback=white,colframe=black]
dpcpp –std=c++17 stdswap.cpp –o stdswap.exe ./stdswap.exe
\end{tcolorbox}

打印结果为:\par

\begin{tcolorbox}[colback=white,colframe=black]
8, 9\\
9, 8
\end{tcolorbox}

图18-8列出了带有“Y”的C++标准API,以表明在编写本文时,这些API已经在用于CPU、GPU和FPGA设备的DPC++内核中进行了测试。空白表示在本书出版时不完全覆盖(不是所有三种设备类型)。在线DPC++手册中也包含了一个表,并将随着时间的推移而更新——DPC++中的库支持将继续扩大它的支持。\par

在DPC++库中,一些C++ std函数是基于设备上相应的内置函数实现,以达到与SYCL版本相同的性能水平。\par

\hspace*{\fill} \par %插入空行
图18-8 库支持CPU/GPU/FPGA覆盖(在本书出版时)
\begin{center}
	\includegraphics[width=1.0\textwidth]{content/chapter-18/images/7}
	\includegraphics[width=1.0\textwidth]{content/chapter-18/images/8}
	\includegraphics[width=1.0\textwidth]{content/chapter-18/images/9}
	\includegraphics[width=1.0\textwidth]{content/chapter-18/images/10}
\end{center}

在libstdc++ (GNU)与gcc 7.4.0和libc++ (LLVM)与clang 10.0和MSVC标准C++库与Microsoft Visual Studio 2017(主机CPU)支持的标准C++ API。\par

在Linux上,GNU libstdc++是DPC++编译器的默认C++标准库,因此不需要编译或链接选项。如果我们想要使用libc++,请使用编译选项-stdlib=libc++ -nodinc++来使用libc++,不要包含系统中的C++ std头文件。DPC++编译器已经在Linux上的DPC++内核中使用libc++进行了验证,但是DPC++运行时需要用libc++而不是libstdc++重新构建。详情请参见https://intel.github.io/llvmdocs/GetStartedGuide.html\#build-dpc-toolchain-with-libc-library。由于这些步骤,libc++不是推荐使用的C++标准库。\par

在FreeBSD上,libc++是默认的标准库,-stdlib=libc++选项不是必需的。更多详情请登录https://libcxx.llvm.org/docs/UsingLibcxx.html。在Windows上,只能使用MSVC c++库。\par

\begin{tcolorbox}[colback=red!5!white,colframe=red!75!black]
为了实现跨架构的可移植性,如果std函数在图18-8中没有标记“Y”,那么在编写设备函数时需要注意是否可移植!
\end{tcolorbox}

\hspace*{\fill} \par %插入空行
\textbf{DPC++的Parallel STL}

Parallel STL是C++标准库算法的一个实现,支持执行策略,如ISO/IEC 14882:2017标准,通常称为C++17。现有的实现还支持Parallelism TS version 2中指定的未排序执行策略,并在C++工作组P1001R1中为下一版本的C++标准提出了该策略。\par

当使用算法和执行策略时,如果没有特定于供应商的C++17标准库实现,则指定命名空间std::execution,否则指定命名空间pstl::execution。\par

对于任何已实现的算法,都可以将seq、unseq、par或par\_unseq中的一个值作为调用算法的第一个参数传递,以指定所需的执行策略。这些策略的含义如下:\par

\begin{table}[]
	\begin{tabular}{|l|l|}
		\hline
		执行策略 & 含义                                                                                                                                            \\ \hline
		seq              & 串行执行                                                                                                                              \\ \hline
		unseq            & \begin{tabular}[c]{@{}l@{}}同步执行SIMD。此策略要求在SIMD中安全执行所提供的所有函数。\end{tabular} \\ \hline
		par              & 由多个线程并行执行。                                                                                                            \\ \hline
		par\_unseq       & 合并了unseq与par的特性                                                                                                                    \\ \hline
	\end{tabular}
\end{table}

对DPC++的Parallel STL进行了扩展,支持使用特殊执行策略的DPC++设备。DPC++执行策略指定了并行STL算法运行的位置和方式,继承了标准C++执行策略。封装了一个SYCL设备或队列,并允许设置可选的内核名称。DPC++执行策略可以与所有支持C++17标准执行策略的算法一起使用。\par

\hspace*{\fill} \par %插入空行
\textbf{DPC++执行策略}

目前,DPC++库只支持并行未排序策略(par\_unseq)。使用DPC++执行策略,有三个步骤:\par

\begin{enumerate}
	\item 在代码中添加\#include <dpstd/execution>。
	\item 通过提供标准策略类型、作为模板参数唯一内核名的类型(可选)和以下构造函数参数之一来创建策略对象:
	\begin{itemize}
		\item SYCL队列
		\item SYCL设备
		\item SYCL设备选择器
		\item 具有不同内核名称的已存在策略对象
	\end{itemize}
	\item 将创建的策略对象传递给Parallel STL算法。
\end{enumerate}

dpstd::execution::default\_policy对象是一个预定义的device\_policy,使用默认的内核名和默认的队列创建。这可以用于创建自定义策略对象,或者在调用算法时直接传递(如果默认选择足够的话)。\par

图18-9显示了使用using命名空间dpstd::execution的示例,引用策略类和函数。\par

\hspace*{\fill} \par %插入空行
图18-9 创建执行策略
\begin{lstlisting}[caption={}]
auto policy_b = 
	device_policy<parallel_unsequenced_policy, class PolicyB> 
		{sycl::device{sycl::gpu_selector{}}};
std::for_each(policy_b, …);

auto policy_c = 
	device_policy<parallel_unsequenced_policy, class PolicyС> 
		{sycl::default_selector{}};
std::for_each(policy_c, …);

auto policy_d = make_device_policy<class PolicyD>(default_policy);
std::for_each(policy_d, …);

auto policy_e = make_device_policy<class PolicyE>(sycl::queue{});
std::for_each(policy_e, …);
\end{lstlisting}

\hspace*{\fill} \par %插入空行
\textbf{FPGA的执行策略}

fpga\_device\_policy是一个DPC++策略类,用于在FPGA硬件上实现性能更好的并行算法。在FPGA硬件或FPGA仿真设备上运行应用程序时,可以使用该策略:\par

\begin{enumerate}
	\item FPGA实际设备运行时设置\_PSTL\_FPGA\_DEVICE宏,以及在FPGA仿真器上运行时设置\_PSTL\_FPGA\_EMU宏。
	\item 在代码中添加\#include <dpstd/ execution>。
	\item 通过为唯一的内核名和展开因子(参见第17章)提供类类型作为模板参数(两个都是可选的)和以下构造函数参数之一来创建策略对象:
	\begin{itemize}
		\item 为FPGA选择器构造的SYCL队列(任何其他设备类型的行为都未定义)
		\item 具有不同内核名称和/或展开因子的FPGA策略对象
	\end{itemize}
	\item 将创建的策略对象传递给Parallel STL算法。
\end{enumerate}

fpga\_device\_policy的默认构造函数创建一个对象,其中包含为FPGA选择器构造的SYCL队列。如果定义了\_PSTL\_FPGA\_EMU,则为FPGA仿真器选择器构造的SYCL队列。\par

dpstd::execution::fpga\_policy是fpga\_device\_policy类的预定义对象,使用默认的内核名称和默认的展开因子创建。使用它来创建定制的策略对象,或者在调用算法时直接使用。\par

图18-10中的代码假设using namespace dpstd::execution;,用于策略,而using namespace sycl;,用于队列和设备选择器。\par

指定策略的展开因子可以在算法的实现中展开循环,默认值为1。了解如何选择一个更好的值,可以回顾第17章。\par

\hspace*{\fill} \par %插入空行
图18-10 FPGA使用执行策略
\begin{lstlisting}[caption={}]
auto fpga_policy_a = fpga_device_policy<class FPGAPolicyA>{};

auto fpga_policy_b = make_fpga_policy(queue{intel::fpga_selector{}});

constexpr auto unroll_factor = 8;
auto fpga_policy_c = 
make_fpga_policy<class FPGAPolicyC, unroll_factor>(fpga_policy);
\end{lstlisting}

\hspace*{\fill} \par %插入空行
\textbf{使用DPC++ Parallel STL}

为了使用DPC++ Parallel STL,需要通过添加以下行集的子集来包含Parallel STL头文件。这些头文件,取决于我们所要使用的算法:\par

\begin{itemize}
	\item \#include <dpstd/algorithm>
	\item \#include <dpstd/numeric>
	\item \#include <dpstd/memory>
\end{itemize}

dpstd::begin和dpstd::end是特殊的辅助函数,允许将SYCL缓冲区传递给Parallel STL算法。这些函数接受SYCL缓冲区,并返回一个未指定类型的满足以下要求的对象:\par

\begin{itemize}
	\item 可拷贝构造,可拷贝赋值,并且支持比较操作符==和!=。
	\item 以下表达式是有效的:a+n, a-n, a-b,其中a和b是该类型的对象,n是一个整数值。
	\item 有一个没有参数的get\_buffer方法。该方法返回传递给dpstd::begin和dpstd::end函数的SYCL缓冲区。
\end{itemize}

要使用这些辅助函数,先将\#include <dpstd/iterators>添加到代码中。图18-11和18-12中的代码,使用std::fill函数作为使用开始/结束帮助器的示例。。\par

\hspace*{\fill} \par %插入空行
图18-11 使用std::fill
\begin{lstlisting}[caption={}]
#include <dpstd/execution>
#include <dpstd/algorithm>
#include <dpstd/iterators>

sycl::queue Q;
sycl::buffer<int> buf { 1000 };

auto buf_begin = dpstd::begin(buf);
auto buf_end = dpstd::end(buf);

auto policy = dpstd::execution::make_device_policy<class fill>( Q );
std::fill(policy, buf_begin, buf_end, 42);
// each element of vec equals to 42
\end{lstlisting}

\begin{tcolorbox}[colback=blue!5!white,colframe=blue!75!black, title=减少主机和设备之间的数据复制]
并行STL算法可以用普通(主机端)迭代器调用,如图18-11中的代码示例所示。\\

本例中,将创建临时SYCL缓冲区,并将数据复制到该缓冲区。设备上的临时缓冲区处理完成后,数据复制回主机。如果可能,建议直接使用现有的SYCL缓冲区,以减少主机和设备之间的数据移动,以及避免创建和销毁缓冲区的不必要的开销。
\end{tcolorbox}

\hspace*{\fill} \par %插入空行
图18-12 使用默认策略的std::fill
\begin{lstlisting}[caption={}]
#include <dpstd/execution>
#include <dpstd/algorithm>

std::vector<int> v( 1000000 );
std::fill(dpstd::execution::default_policy, v.begin(), v.end(), 42);
// each element of vec equals to 42
\end{lstlisting}

图18-13显示了一个示例,该示例对提供的搜索序列中的每个值执行输入序列的二进制搜索。作为搜索序列的第i个元素的结果,将一个指示搜索值是否在输入序列中找到的布尔值,赋给结果序列的第i个元素。该算法返回的迭代器指向赋值序列的最后一个元素的下一个位置。该算法假定输入序列已提供了比较排序。如果没有提供比较器,则使用使用操作符<进行元素比较的函数对象。\par

前面描述的复杂性强调了应该尽可能利用库函数,而不是自己编写算法实现,这可能需要大量的调试和调优时间。可利用的库作者通常都是编码的设备架构的专家,他们可能能看到一些非公开的信息,所以当优化的库可用时,应该先使用它们。\par

图18-13所示的代码示例演示了使用DPC++并行STL算法时的三个步骤:\par

\begin{itemize}
	\item 创建DPC++的迭代器。
	\item 从现有策略创建命名策略。
	\item 调用并行算法
\end{itemize}

图18-13中的示例使用dpstd::binary\_search算法根据我们的设备选择在CPU、GPU或FPGA上执行二叉树搜索。\par

\hspace*{\fill} \par %插入空行
图18-13 使用binary\_search
\begin{lstlisting}[caption={}]
#include <dpstd/execution>
#include <dpstd/algorithm>
#include <dpstd/iterator>

buffer<uint64_t, 1> kB{ range<1>(10) };
buffer<uint64_t, 1> vB{ range<1>(5) };
buffer<uint64_t, 1> rB{ range<1>(5) };

accessor k{kB};
accessor v{vB};

// create dpc++ iterators
auto k_beg = dpstd::begin(kB);
auto k_end = dpstd::end(kB);
auto v_beg = dpstd::begin(vB);
auto v_end = dpstd::end(vB);
auto r_beg = dpstd::begin(rB);

// create named policy from existing one
auto policy = dpstd::execution::make_device_policy<class bSearch>
	(dpstd::execution::default_policy);

// call algorithm
dpstd::binary_search(policy, k_beg, k_end, v_beg, v_end, r_beg);

// check data
accessor r{rB};
if ((r[0] == false) && (r[1] == true) && 
(r[2] == false) && (r[3] == true) && (r[4] == true)) {
	std::cout << "Passed.\nRun on "
	<< policy.queue().get_device().get_info<info::device::name>()
	<< "\n";
} else
	std::cout << "failed: values do not match.\n";
\end{lstlisting}

\hspace*{\fill} \par %插入空行
\textbf{并行STL与USM}

下面的例子描述了并行STL算法与USM结合使用的两种方法:\par

\begin{itemize}
	\item 通过USM指针
	\item 通过USM分配器
\end{itemize}

如果有一个USM分配器,可以将指向分配开始和结束的指针传递给并行算法。重要的是要确保执行策略和分配本身是为相同的队列或上下文创建的,以避免在运行时出现未定义行为。\par

如果相同的分配要由多个算法处理,可以使用有序队列,或者显式地等待每个算法完成后再在下一个算法中使用相同的分配(这是使用USM时的操作顺序)。也需要等待操作完成后才能访问主机上的数据,如图18-14所示。\par

或者,也可以使用std::vector和USM分配器,如图18-15所示。\par

\hspace*{\fill} \par %插入空行
图18-14 使用带有USM指针的并行STL
\begin{lstlisting}[caption={}]
#include <dpstd/execution>
#include <dpstd/algorithm>

sycl::queue q;
const int n = 10;
int* d_head = static_cast<int*>(
	sycl::malloc_device(n * sizeof(int),
						q.get_device(), 
						q.get_context()));
						
std::fill(dpstd::execution::make_device_policy(q),
		  d_head, d_head + n, 78);
q.wait();

sycl::free(d_head, q.get_context());
\end{lstlisting}

\hspace*{\fill} \par %插入空行
图18-15 使用并行STL与USM分配器
\begin{lstlisting}[caption={}]
#include <dpstd/execution>
#include <dpstd/algorithm>

sycl::queue Q;
const int n = 10;
sycl::usm_allocator<int, sycl::usm::alloc::shared> 
						 alloc(Q.get_context(), Q.get_device());
std::vector<int, decltype(alloc)> vec(n, alloc);

std::fill(dpstd::execution::make_device_policy(Q), 
							vec.begin(), vec.end(), 78);
Q.wait();
\end{lstlisting}

\hspace*{\fill} \par %插入空行
\textbf{错误处理与DPC++的执行策略}

如第5章所述,DPC++错误处理模型支持两种类型的错误。对于同步错误,运行时抛出异常,而异步错误只在程序执行期间的指定时间在用户提供的错误处理程序中处理。\par

对于使用DPC++策略执行的并行STL算法,处理所有错误(同步或异步)是调用者的责任。具体地说\par

\begin{itemize}
	\item 算法不会显式抛出异常。
	\item 运行时在主机CPU上抛出的异常(包括DPC++同步异常),并传递给调用者。
	\item Parallel STL不处理DPC++异步错误,因此必须由应用程序处理。
\end{itemize}

要处理DPC++异步错误,必须用错误处理程序对象创建与DPC++策略关联的队列。预定义的策略对象(default\_policy和其他)没有错误处理程序,因此如果需要处理异步错误,应该创建自己的策略。\par

































