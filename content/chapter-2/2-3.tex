
当我们不关心设备代码将在哪里运行时,可以让运行时库帮我们选择。这种自动选择的目的是,当我们还不关心选择了什么设备时,可以很容易地开始编写和运行代码。这种设备选择没有考虑要运行的代码特性,所以是一个任意的选择,这个选择很可能不是最优。\par

讨论设备的选择(即使是实现为选择的设备)之前,我们应该首先讨论与设备交互的机制:\textbf{队列}。\par

\hspace*{\fill} \par %插入空行
\textbf{队列}

队列是一种抽象,将操作提交给它在单个设备上执行。图2-3和图2-4给出了\textit{queue}类的定义。入队的通常是数据并行的计算,当我们想要更多的控制权时,也可以使用其他命令,如手动控制数据移动。提交到队列的工作需要满足在运行时跟踪的先决条件(例如输入数据的可用性)之后进行执行。这些先决条件在第3章和第8章中有介绍。\par

\hspace*{\fill} \par %插入空行
图2-3 简化的queue类
\begin{lstlisting}[caption={}]
class queue {
public:
	// Create a queue associated with the default device
	queue(const property_list = {});
	queue(const async_handler&, 
		  const property_list = {});
	
	// Create a queue associated with an explicit device
	// A device selector may be used in place of a device
	queue(const device&, const property_list = {});
	queue(const device&, const async_handler&, 
	      const property_list = {});
	
	// Create a queue associated with a device in a specific context
	// A device selector may be used in place of a device
	queue(const context&, const device&, 
		  const property_list = {});
	queue(const context&, const device&, 
		  const async_handler&, 
		  const property_list = {});
};
\end{lstlisting}

\hspace*{\fill} \par %插入空行
图2-4 简化了queue类中的关键成员函数
\begin{lstlisting}[caption={}]
class queue {
public:
	// Submit a command group to this queue.
	// The command group may be a lambda or functor object.
	// Returns an event representation the action 
	// performed in the command group.
	template <typename T>
	event submit(T);
	
	// Wait for all previously submitted actions to finish executing.
	void wait();
	
	// Wait for all previously submitted actions to finish executing.
	// Pass asynchronous exceptions to an async_handler if one was provided.
	void wait_and_throw();
};
\end{lstlisting}	
	
队列绑定到单个设备,绑定发生在队列的构造过程中。重要的是,提交给队列的工作是在该队列绑定的单个设备上执行的。队列不能映射到设备集合,因为这将导致不明确应该在哪个设备上执行工作。同样,队列也不能将提交给它的工作分散到多个设备上。相反,队列和提交给队列的工作将在其上执行的设备之间有一个明确的映射,如图2-5所示。\par

\hspace*{\fill} \par %插入空行
图2-5 队列绑定到单个设备,提交到队列中的工作在相应设备上执行
\begin{center}
	\includegraphics[width=1.\textwidth]{content/chapter-2/images/4}
\end{center}

程序中可以以希望的程序架构或编程风格创建多个队列,例如:可以创建多个队列,让每个队列与不同的设备绑定,或者让宿主程序中的不同线程使用。多个不同的队列可以绑定到单个设备上,比如GPU,提交到这些不同队列的工作将组合在设备上执行,如图2-6所示。相反,正如我们前面提到的,队列不能绑定到多个设备,因为在请求执行操作的位置上不能有任何歧义。例如,如果想要一个能够跨多个设备加载平衡工作的队列,可以在代码中创建这个对象。\par

\hspace*{\fill} \par %插入空行
图2-6 多个队列可以绑定到单个设备
\begin{center}
	\includegraphics[width=1.\textwidth]{content/chapter-2/images/5}
\end{center}

因为队列绑定到特定的设备,所以提交给队列的操作是将相应工作在设备上执行的常用方法。构造队列时选择设备是通过设备选择器和\textit{device\_selector}类实现的。\par

\hspace*{\fill} \par %插入空行
\textbf{绑定一个队列到一个设备(任何设备都可以)}

图2-7是一个没有指定队列绑定设备的示例。queue的构造函数没有任何参数(如图2-7所示),只是选择一些可用的设备。SYCL保证至少有一个设备始终可用,即主机设备。主机设备可以运行内核代码,它是执行主机程序的处理器,所以总是存在。\par

\hspace*{\fill} \par %插入空行
图2-7 通过队列的隐式构造,选择默认设备
\begin{lstlisting}[caption={}]
#include <CL/sycl.hpp>
#include <iostream>
using namespace sycl;

int main() {
	// Create queue on whatever default device that the implementation
	// chooses. Implicit use of the default_selector. 
	queue Q;
	
	std::cout << "Selected device: " <<
	Q.get_device().get_info<info::device::name>() << "\n";
	
	return 0;
}

Possible Output:
Device: SYCL host device
\end{lstlisting}	

使用queue的构造函数是开始开发和启动,并运行设备代码的最简单方法。对于绑定到队列的设备的选择,因为它与我们的应用程序相关,可以增加更多的控制。\par













