本节中,我们将深入讨论一个主题,其会给SYCL的新用户带来困惑,以及新SYCL开发人员会遇到(以我们的经验)的第一个bug。\par

简单地说,当从主机内存分配(例如,数组或向量)创建缓冲区时,主机不能直接访问。缓冲区的整个生命周期内,缓冲区拥有在构造时传入的主机内存。很少有机制允许在缓冲区还存在时访问主机内存(例如,缓冲区互斥),但是这些高级特性对调试早期bug没有帮助。\par

\begin{tcolorbox}[colback=red!5!white,colframe=red!75!black]
如果使用主机内存构造缓冲区,不能直接访问主机内存,直到缓冲区销毁!当缓冲区处于活动状态时,其会占有相应内存。
\end{tcolorbox}

当主机程序访问主机内存,而缓冲区仍然拥有该内存时,会出现一个错误,因为不知道缓冲区的类型。如果数据是错误的,不要感到惊讶。如第3章和第8章所述,SYCL是围绕异步任务图构建的。尝试使用任务图操作输出数据之前,需要确保已经到达了同步点,并使数据对主机可用。缓冲区销毁和主机访问器的创建都会触发同步。\par

图13-6展示了代码的模式,通过关闭定义缓冲区的作用域来销毁缓冲区。通过销毁缓冲区,可以给缓冲区构造函数的主机内存,从而可以安全地读取内核结果。\par

\hspace*{\fill} \par %插入空行
图13-6 从主机内存创建通用模式的缓冲区
\begin{lstlisting}[caption={}]
constexpr size_t N = 1024;

// Set up queue on any available device
queue q;

// Create host containers to initialize on the host
std::vector<int> in_vec(N), out_vec(N);

// Initialize input and output vectors
for (int i=0; i < N; i++) in_vec[i] = i;
std::fill(out_vec.begin(), out_vec.end(), 0);

// Nuance: Create new scope so that we can easily cause
// buffers to go out of scope and be destroyed
{
	
	// Create buffers using host allocations (vector in this case)
	buffer in_buf{in_vec}, out_buf{out_vec};
	
	// Submit the kernel to the queue
	q.submit([&](handler& h) {
		accessor in{in_buf, h};
		accessor out{out_buf, h};
		
		h.parallel_for(range{N}, [=](id<1> idx) {
			out[idx] = in[idx];
		});
	});

	// Close the scope that buffer is alive within! Causes
	// buffer destruction which will wait until the kernels
	// writing to buffers have completed, and will copy the
	// data from written buffers back to host allocations (our
	// std::vectors in this case). After the buffer destructor
	// runs, caused by this closing of scope, then it is safe
	// to access the original in_vec and out_vec again!
}

// Check that all outputs match expected value
// WARNING: The buffer destructor must have run for us to safely
// use in_vec and out_vec again in our host code. While the buffer
// is alive it owns those allocations, and they are not safe for us
// to use! At the least they will contain values that are not up to
// date. This code is safe and correct because the closing of scope
// above has caused the buffer to be destroyed before this point
// where we use the vectors again.
for (int i=0; i<N; i++) 
	std::cout << "out_vec[" << i << "]=" << out_vec[i] << "\n";
\end{lstlisting}

将缓冲区与现有主机内存关联有两个常见的原因,如图13-6所示:\par

\begin{enumerate}
	\item 简化缓冲区的初始化。可以从已经初始化的主机内存(或应用程序的其他部分)中构造缓冲区。
	\item 减少输入的字符,因为用“\}”关闭作用域比创建缓冲区的host\_accessor更简洁(更容易出错)。
\end{enumerate}

如果使用主机内存转储或验证内核的输出值,需要将缓冲区分配放入作用域(或其他作用域)中,这样就可以控制缓冲区的销毁了。然后,必须确保访问主机内存,在获得内核输出之前销毁缓冲区。图13-6显展了这方面的正确实现,而图13-7展示了一个错误实现,即仍在缓冲区活动时访问输出。\par

\begin{tcolorbox}[colback=red!5!white,colframe=red!75!black]
高级用户可能更喜欢使用缓冲区销毁,将结果数据从内核返回到主机内存分配内存中。对于大多数用户,特别是新手开发者,建议使用作用域内的主机访问器。
\end{tcolorbox}

\hspace*{\fill} \par %插入空行
图13-7 常见错误:缓冲区生命周期内直接从主机分配内存中读取数据
\begin{lstlisting}[caption={}]
constexpr size_t N = 1024;

// Set up queue on any available device
queue q;

// Create host containers to initialize on the host
std::vector<int> in_vec(N), out_vec(N);

// Initialize input and output vectors
for (int i=0; i < N; i++) in_vec[i] = i;
std::fill(out_vec.begin(), out_vec.end(), 0);

// Create buffers using host allocations (vector in this case)
buffer in_buf{in_vec}, out_buf{out_vec};

// Submit the kernel to the queue
q.submit([&](handler& h) {
	accessor in{in_buf, h};
	accessor out{out_buf, h};
	
	h.parallel_for(range{N}, [=](id<1> idx) {
		out[idx] = in[idx];
	});
});

// BUG!!! We're using the host allocation out_vec, but the buffer out_buf
// is still alive and owns that allocation! We will probably see the
// initialiation value (zeros) printed out, since the kernel probably
// hasn't even run yet, and the buffer has no reason to have copied
// any output back to the host even if the kernel has run.
for (int i=0; i<N; i++)
	std::cout << "out_vec[" << i << "]=" << out_vec[i] << "\n";
\end{lstlisting}

\begin{tcolorbox}[colback=red!5!white,colframe=red!75!black]
请使用主机访问器,而不是直接访问内存,特别是在开始时。
\end{tcolorbox}

为了避免这些错误,建议在开始使用SYCL和DPC++时使用主机访问器,而不是直接访问内存。主机访问器提供了对来自主机的缓冲区的访问。当构造完成时,就可以保证之前对缓冲区的任何写入(例如,创建host\_accessor之前提交的内核)都已经执行并且可见。本书混合使用了这两种风格(即,主机访问器和传递给缓冲区构造函数的主机分配),以便熟悉这两种风格。开始时,使用主机访问器不容易出错。图13-8展示了如何使用主机访问器从内核读取输出,而不破坏缓冲区。\par

\hspace*{\fill} \par %插入空行
图13-8 建议:使用主机访问器读取内核结果
\begin{lstlisting}[caption={}]
constexpr size_t N = 1024;

// Set up queue on any available device
queue q;

// Create host containers to initialize on the host
std::vector<int> in_vec(N), out_vec(N);

// Initialize input and output vectors
for (int i=0; i < N; i++) in_vec[i] = i;
std::fill(out_vec.begin(), out_vec.end(), 0
);
// Create buffers using host allocations (vector in this case)
buffer in_buf{in_vec}, out_buf{out_vec};

// Submit the kernel to the queue
q.submit([&](handler& h) {
	accessor in{in_buf, h};
	accessor out{out_buf, h};
	
	h.parallel_for(range{N}, [=](id<1> idx) {
		out[idx] = in[idx];
	});
});

// Check that all outputs match expected value
// Use host accessor! Buffer is still in scope / alive
host_accessor A{out_buf};

for (int i=0; i<N; i++) std::cout << "A[" << i << "]=" << A[i] << "\n";
\end{lstlisting}

只要缓冲区没销毁,就可以使用主机访问器,例如:在缓冲区生命周期的两端,用于初始化缓冲区内容和从内核读取结果,如图13-9所示。\par

\hspace*{\fill} \par %插入空行
图13-9 建议:使用主机访问器进行缓冲区初始化和读取结果
\begin{lstlisting}[caption={}]
constexpr size_t N = 1024;

// Set up queue on any available device
queue q;

// Create buffers of size N
buffer<int> in_buf{N}, out_buf{N};

// Use host accessors to initialize the data
{ // CRITICAL: Begin scope for host_accessor lifetime!
	host_accessor in_acc{ in_buf }, out_acc{ out_buf };
	for (int i=0; i < N; i++) {
		in_acc[i] = i;
		out_acc[i] = 0;
	}
} // CRITICAL: Close scope to make host accessors go out of scope!

// Submit the kernel to the queue
q.submit([&](handler& h) {
	accessor in{in_buf, h};
	accessor out{out_buf, h};
	
	h.parallel_for(range{N}, [=](id<1> idx) {
		out[idx] = in[idx];
	});
});

// Check that all outputs match expected value
// Use host accessor! Buffer is still in scope / alive
host_accessor A{out_buf};

for (int i=0; i<N; i++) std::cout << "A[" << i << "]=" << A[i] << "\n";
\end{lstlisting}

最后要提到的是,主机访问器有时会在应用程序中引发错误,因为它们也有生命周期。当缓冲区的host\_accessor在生命周期中,运行时将不允许任何设备使用该缓冲区!运行时不会分析主程序来确定什么时候可以访问主机访问器,所以主机程序访问缓冲区的唯一方法是运行host\_accessor析构函数。如图13-10所示,如果主程序正在等待内核运行(例如,queue::wait()或获取另一个主机访问器),如果DPC++运行时正在等待之前的主机访问器销毁,在销毁之后才能操作使用缓冲区的内核,这可能会导致应用程序挂起等待。\par

\begin{tcolorbox}[colback=red!5!white,colframe=red!75!black]
使用主机访问器时,请确保内核或其他主机访问器不再需要解锁缓冲区时销毁它们。
\end{tcolorbox}

\hspace*{\fill} \par %插入空行
图13-10 host\_accessors使用不当是会挂起程序
\begin{lstlisting}[caption={}]
constexpr size_t N = 1024;

// Set up queue on any available device
queue q;

// Create buffers using host allocations (vector in this case)
buffer<int> in_buf{N}, out_buf{N};

// Use host accessors to initialize the data
host_accessor in_acc{ in_buf }, out_acc{ out_buf };
for (int i=0; i < N; i++) {
	in_acc[i] = i;
	out_acc[i] = 0;
}

// BUG: Host accessors in_acc and out_acc are still alive!
// Later q.submits will never start on a device, because the
// runtime doesn't know that we've finished accessing the
// buffers via the host accessors. The device kernels
// can't launch until the host finishes updating the buffers,
// since the host gained access first (before the queue submissions).
// This program will appear to hang! Use a debugger in that case.

// Submit the kernel to the queue
q.submit([&](handler& h) {
	accessor in{in_buf, h};
	accessor out{out_buf, h};
	
	h.parallel_for(range{N}, [=](id<1> idx) {
		out[idx] = in[idx];
	});
});

std::cout <<
	"This program will deadlock here!!! Our host_accessors used\n"
	" for data initialization are still in scope, so the runtime won't\n"
	" allow our kernel to start executing on the device (the host could\n"
	" still be initializing the data that is used by the kernel). "
	" The next line\n of code is acquiring a host accessor for"
	" the output, which will wait for\n the kernel to run first. "
	" Since in_acc and out_acc have not been\n"
	" destructed, the kernel is not safe for the runtime to run, "
	" and we deadlock.\n";
	
// Check that all outputs match expected value
// Use host accessor! Buffer is still in scope / alive
host_accessor A{out_buf};

for (int i=0; i<N; i++) std::cout << "A[" << i << "]=" << A[i] << "\n";
\end{lstlisting}












































