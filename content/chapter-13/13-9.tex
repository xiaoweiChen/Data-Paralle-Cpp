Migrating CUDA code to SYCL or DPC++ is not covered in detail in this book. There are tools and resources available that explore doing this. Migrating CUDA code is relatively straightforward since it is a kernel-based approach to parallelism. Once written in SYCL or DPC++, the new program is enhanced by its ability to target more devices than supported by CUDA alone. The newly enhanced program can still be targeted to NVIDIA GPUs using SYCL compilers with NVIDIA GPU support.\par

Migrating to SYCL opens the door to the diversity of devices supported by SYCL, which extends far beyond just GPUs.\par

When using the DPC++ Compatibility Tool, the --report-type=value option provides very useful statistics about the migrated code. One of the reviewers of this book called it a “beautiful flag provided by Intel dpct.” The --in-root option can prove very useful when migrating CUDA code depending on source code organization of a project.\par

To learn more about CUDA migration, two resources are a good place to start:\par

\begin{itemize}
	\item Intel’s DPC++ Compatibility Tool transforms CUDA applications into DPC++ code (tinyurl.com/CUDAtoDPCpp).
	\item Codeplay tutorial “Migrating from CUDA to SYCL” 	(tinyurl.com/codeplayCUDAtoSYCL).
\end{itemize}















