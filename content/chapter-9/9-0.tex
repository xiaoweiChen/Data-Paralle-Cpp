\begin{center}
	\includegraphics[width=0.3\textwidth]{content/chapter-9/images/1}
\end{center}

In Chapter 4, we discussed ways to express parallelism, either using basic data-parallel kernels, explicit ND-range kernels, or hierarchical parallel kernels. We discussed how basic data-parallel kernels apply the same operation to every piece of data independently. We also discussed how explicit ND-range kernels and hierarchical parallel kernels divide the execution range into work-groups of work-items.\par

In this chapter, we will revisit the question of how to break up a problem into bite-sized chunks in our continuing quest to Think Parallel. This chapter provides more detail regarding explicit ND-range kernels and hierarchical parallel kernels and describes how groupings of work-items may be used to improve the performance of some types of algorithms. We will describe how groups of work-items provide additional guarantees for how parallel work is executed, and we will introduce language features that support groupings of work-items. Many of these ideas and concepts will be important when optimizing programs for specific devices in Chapters 15, 16, and 17 and to describe common parallel patterns in Chapter 14.\par