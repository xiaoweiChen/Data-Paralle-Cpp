In Chapter 3, we discussed data management and ordering the uses of data. That chapter described the key abstraction behind graphs in DPC++: dependences. Dependences between kernels are fundamentally based on what data a kernel accesses. A kernel needs to be certain that it reads the correct data before it can compute its output.\par

We described the three types of data dependences that are important for ensuring correct execution. The first, Read-after-Write (RAW), occurs when one task needs to read data produced by a different task. This type of dependence describes the flow of data between two kernels. The second type of dependence happens when one task needs to update data after another task has read it. We call that type of dependence a Write-afterRead (WAR) dependence. The final type of data dependence occurs when two tasks try to write the same data. This is known as a Write-after-Write (WAW) dependence.\par

Data dependences are the building blocks we will use to build graphs. This set of dependences is all we need to express both simple linear chains of kernels and large, complex graphs with hundreds of kernels with elaborate dependences. No matter which types of graph a computation needs, DPC++ graphs ensure that a program will execute correctly based on the expressed dependences. However, it is up to the programmer to make sure that a graph correctly expresses all the dependences in a program.\par