我们将讨论的最后一个主题是,如何同步图和主机执行。我们在本章中已经涉及到这一点,但现在我们将研究程序实现这一点的所有方法。\par

主机同步的第一个方法是在前面的许多示例中使用过的方法:等待队列。队列对象有两个方法,wait和wait\_and\_throw,阻塞执行,直到提交到队列的每个命令组都完成。这是一个非常简单的方法,可以处理许多常见的情况。然而,需要指出的是,这种方法是粒度非常粗。如果需要更细粒度的同步,将使用另一种方法。\par

主机同步的下一种方法是对事件进行同步。这比在队列上同步更灵活,因为它只允许程序在特定的操作或命令组上同步。这可以通过在事件上调用wait方法或在事件类上调用静态方法wait来完成,该方法可以接受一个事件组。\par

我们已经看到了图8-5和图8-8中使用的下一个方法:主机访问器。主机访问器执行两个功能。首先,使主机上的数据可用。其次,通过在当前访问的图和主机之间定义一个新的依赖关系来与主机同步。这可以确保复制回主机的数据是图计算的正确结果。但是,如果缓冲区是由主机内存构造的,那么这个原始内存不能保证数值的一致性。\par

注意,主机访问器是阻塞的。在数据可用之前,主机上的执行可能不会在创建主机访问器之后继续。同样,当主机访问器存在并保持其数据可用时,不能在设备上使用缓冲区。一种常见的模式是在C++作用域中创建主机访问器,以便在不再需要主机访问器时释放数据。这是下一种主机同步的方法。\par

DPC++中的某些对象在销毁和调用其析构函数时具有特殊行为。我们刚刚了解了主机访问器如何使数据在被销毁之前保持在主机上。当缓冲区和图像被销毁或离开生命周期时,也有特殊的行为。当缓冲区销毁时,将等待所有使用该缓冲区的命令组完成执行。当任何内核或内存操作不再使用缓冲区,运行时必须将数据复制回主机。如果缓冲区使用宿主指针初始化,或者宿主指针传递给方法set\_final\_data,则会发生复制。然后,运行库将复制该缓冲区的数据,并在销毁对象之前更新主机内存。\par

与主机同步的最后一个选项涉及在第7章中首先描述的一个不常见的特性。回想一下,缓冲区对象的构造函数可选地接受属性列表。创建缓冲区时,传递的有效属性是use\_mutex。当以这种方式创建缓冲区时,增加了一个要求,即该缓冲区拥有的内存可以与主机程序共享。对该内存的访问由互斥锁控制,当可以安全地访问与缓冲区共享的内存时,主机能够获得互斥锁。如果无法获得锁,用户可能需要将内存移动操作排队,以便与主机同步数据。这种用法非常小众,在大多数DPC++应用程序中不太可能看到。\par

