我们已经展示了许多使用handler类的方法。图3-17和图3-18更详细地解释了这个非常重要的类的关键成员。我们目前还没有使用所有的成员,后续会对它们进行使用。这里对他们进行展示。\par

另一个queue类在第2章的末尾也有类似的解释,在线oneAPI DPC++语言参考提供了对这两个类更详细的解释。\par

\hspace*{\fill} \par %插入空行
图3-17 简化了handler类的非访问器成员
\begin{lstlisting}[caption={}]
class handler {
	...
	// Specifies event(s) that must be complete before the action
	// defined in this command group executes.
	void depends_on({event / std::vector<event> & });
	
	// Enqueues a memset operation on the specified pointer.
	// Writes the first byte of Value into Count bytes.
	// Returns an event representing this operation.
	event memset(void *Ptr, int Value, size_t Count);
	
	// Enqueues a memcpy from Src to Dest.
	// Count bytes are copied.
	// Returns an event representing this operation.
	event memcpy(void *Dest, const void *Src, size_t Count);
	
	// Submits a kernel of one work-item for execution.
	// Returns an event representing this operation.
	template <typename KernelName, typename KernelType>
	event single_task(KernelType KernelFunc);
	
	// Submits a kernel with NumWork-items work-items for execution.
	// Returns an event representing this operation.
	template <typename KernelName, typename KernelType, int Dims>
	event parallel_for(range<Dims> NumWork-items, KernelType KernelFunc);
	
	// Submits a kernel for execution over the supplied nd_range.
	// Returns an event representing this operation.
	template <typename KernelName, typename KernelType, int Dims>
	event parallel_for(nd_range<Dims> ExecutionRange, KernelType KernelFunc);
	...
};
\end{lstlisting}


\hspace*{\fill} \par %插入空行
图3-18 简化定义handler类的访问器成员
\begin{lstlisting}[caption={}]
class handler {
	...
	// Specifies event(s) that must be complete before the action
	// Copy to/from an accessor.
	// Valid combinations:
	// Src: accessor, Dest: shared_ptr
	// Src: accessor, Dest: pointer
	// Src: shared_ptr Dest: accessor
	// Src: pointer Dest: accessor
	// Src: accesssor Dest: accessor
	template <typename T_Src, typename T_Dst, 
			int Dims, access::mode AccessMode,
			access::target AccessTarget,
			access::placeholder IsPlaceholder = 
			access::placeholder::false_t>
	void copy(accessor<T_Src, Dims, AccessMode, 
			AccessTarget, IsPlaceholder> Src,
			shared_ptr_class<T_Dst> Dst);
	void copy(shared_ptr_class<T_Src> Src,
			accessor<T_Dst, Dims, AccessMode, 
				AccessTarget, IsPlaceholder> Dst);
	void copy(accessor<T_Src, Dims, AccessMode, 
				AccessTarget, IsPlaceholder> Src,
				T_Dst *Dst);
	void copy(const T_Src *Src,
			accessor<T_Dst, Dims, AccessMode, 
				AccessTarget, IsPlaceholder> Dst);
	template <
		typename T_Src, int Dims_Src, 
		access::mode AccessMode_Src,
		access::target AccessTarget_Src, 
		typename T_Dst, int Dims_Dst,
		access::mode AccessMode_Dst, 
		access::target AccessTarget_Dst,
		access::placeholder IsPlaceholder_Src = 
			access::placeholder::false_t,
		access::placeholder IsPlaceholder_Dst = 
			access::placeholder::false_t>
	void copy(accessor<T_Src, Dims_Src, AccessMode_Src, 
					AccessTarget_Src, IsPlaceholder_Src>
			Src,
			accessor<T_Dst, Dims_Dst, AccessMode_Dst, 
					AccessTarget_Dst, IsPlaceholder_Dst>
			Dst);
			
	// Provides a guarantee that the memory object accessed by the accessor
	// is updated on the host after this action executes.
	template <typename T, int Dims, 
			access::mode AccessMode,
			access::target AccessTarget,
			access::placeholder IsPlaceholder =
				access::placeholder::false_t>
	void update_host(accessor<T, Dims, AccessMode, 
					AccessTarget, IsPlaceholder> Acc);
	...
};
\end{lstlisting}













